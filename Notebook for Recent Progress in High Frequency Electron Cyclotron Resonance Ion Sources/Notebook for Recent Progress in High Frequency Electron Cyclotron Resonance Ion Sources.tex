\documentclass[fontset=windows]{article}

% Language setting
% Replace `english' with e.g. `spanish' to change the document language
\usepackage[english]{babel}

% Set page size and margins
% Replace `letterpaper' with`a4paper' for UK/EU standard size
\usepackage[letterpaper,top=2cm,bottom=2cm,left=3cm,right=3cm,marginparwidth=1.75cm]{geometry}
\usepackage{caption2}
\usepackage{subfigure}
\usepackage{float}

% Useful packages
\usepackage{amsmath}
\usepackage{graphicx}
\graphicspath{{Figures/}}%文章所用图片在当前目录下的 Figures目录
\usepackage[colorlinks=true, allcolors=blue]{hyperref}
\usepackage{ctex}
\usepackage{bm}
\usepackage{gensymb}
\renewcommand{\figurename}{图}
\renewcommand{\tablename}{表}
\newtheorem{defination}{定义}[section]
\newtheorem{theorem}{定理}[section]
\newtheorem{lemma}[theorem]{引理}
\newtheorem{corollary}[theorem]{推论}

\title{ECR离子源阅读笔记}
\author{X.Y. Wang}

\begin{document}
\maketitle

\begin{abstract}
Recent Progress in High Frequency Electron Cyclotron Resonance Ion Sources 文章阅读笔记,争取在4月1日前完成。
\end{abstract}

\section{ECRIS中的微波耦合}
尽管ECRIS已经实现了很高的性能,但是微波馈入模式仍采用很传统的设计。很多的ECRIS工作在微波波长和ECR弧腔尺寸接近的情况,比如14 GHz的常温ECRIS,它的弧腔直径是70 mm,而微波波长为20 mm。这种边界效应主导的情况下,通常认为ECR弧腔是一个多模腔,其中包含很多模式的微波。

\subsection{微观描述}
首先,我们考虑单粒子近似下,电子与高频微波的作用。热等离子体被限制在共振面内,电子沿着磁力线穿梭在两个磁镜峰之间,来回往复运动,并且在经过共振面时获得能量。
注入的微波对等离子体起到了随机加热的效应,在相互作用过程中,电子获得的能量为$ \delta E = v \delta p $,此时微波损失的能量为$\hbar \omega$。于是有关系:
\begin{equation}
\delta E-\hbar\omega=0=v_\perp \delta p_\perp +v_\parallel \delta p_\parallel-\hbar\omega
\label{e1}
\end{equation}

当电子获得一个冲量$\delta p_\parallel$平行于磁力线时,微波将损失相同的动量$\hbar k_\parallel=\hbar k$。
此时公式(\ref{e1})变为
\begin{equation}
v_\perp \delta p_\perp +v_\parallel \delta p_\parallel(1-\frac{v_\phi}{v_\parallel})=0
\label{e2}
\end{equation}
其中$v_\phi=\frac{\omega}{k_\parallel}$是微波的相速度,$v_\perp$和$p_\perp$分别是电子的垂直速度和垂直动量。

如果电子的速度远小于微波的相速度,那么微波将主要改变电子垂直方向的能量。然而当电子速度远大于波速时,那么微波相互作用将变为角扩散过程,公式(\ref{e2})变为
\begin{equation}
v_\perp \delta p_\perp +v_\parallel \delta p_\parallel=0
\label{e3}
\end{equation}
但是\textbf{角扩散过程}会使电子进入到损失锥内。在ECRIS中,电子速度将小于波速,而大于波速的电子将通过损失锥丢失。

或者说,电子在沿着磁力线进行往复运动的时候,如果存在一个垂直于磁力线的电场$E$,那么电子每次经过ECR共振区时,都会获得垂直方向的冲量;如果存在一个平行于磁力线的电场,那么电子每次经过ECR共振区时,都会获得平行方向的冲量,这时电子可能会进入到损失锥里。
我们所说的平行,是指平行于轴线上的磁场,因为大量的热等离子体在轴线上产生。所以,微波是沿轴线平行入射的,产生垂直于轴线的电场。

当射频微波入射到ECRIS中,微波能量将与等离子体波进行耦合,等离子体波会将微波功率传递到ECR共振面来加热电子。

在大多数ECRIS中,真空的波长$\lambda_0$是小于等离子体的尺寸。这些等离子体波可能会参与以下作用:

\begin{enumerate}
    \item [a.]在平行传播或者准平行传播中(波矢$k$平行于磁场$B$),将存在两种情况:
    \begin{enumerate}
        \item[1.]呼啸波(电场圆极化,并且垂直于波矢$k$)$\omega_{rf}<\omega_{ce}$。在递减的磁场中,共振点为$\omega_{rf}\approx\omega_{ce}$。\
        \item[2.]对于高频波,也被称为脉泽波,此时$\omega_{rf}>\omega_{ce}$,其中
        \begin{equation}
            \omega_{pe}=(\frac{n_ee^2}{\epsilon_0m_e})^{1/2}
        \end{equation}
        此时等离子体频率为$f_{pe}\approx9000\sqrt{n_e}$($f_{pe}$的单位为Hz,$n_e$的单位为cm$^{-3}$);回旋频率为
        \begin{equation}
            \omega_{ce}=\frac{eB}{m_e}
        \end{equation}
        回旋频率为$f_{ce}=28\times10^9/\text{T}$($f_{ce}$的单位为Hz)
        此时R波的频率为
        \begin{equation}
            \omega_R=\frac{(4\omega_{pe}^2+\omega_{ce}^2)^{1/2}+\omega_{ce}}{2} \text{  (right wave)}
        \end{equation}
    \end{enumerate}
    \item[b.]对于垂直或者准垂直传播的微波(波矢$k$垂直于磁场$B$):
    \begin{enumerate}
        \item [1.]O波,是电场线极化波,垂直于波矢$k$,平行于磁场$B$。
        \item [2.]X波,是电场椭圆极化波,在垂直于磁场$B$的平面内
        \begin{equation}
            \omega_{UH}=(\omega_{ce}^2+\omega_{pe}^2)^{1/2}\text{  (upper hybrid frequency)}
        \end{equation}
        \begin{equation}
            \omega_L=\frac{(4\omega_{pe}^2+\omega_{ce}^2)^{1/2}-\omega_{ce}}{2}\text{  (left wave)}
        \end{equation}
        微波在$\omega_{ce}$和$\omega_{UH}$频率下共振,在$\omega_{pe}$、$\omega_{R}$和$\omega_{L}$频率下截止。
        当$\omega_{pe}\le\omega_{rf}$时,存在ECRIS的经验公式为:
        \begin{equation}
            (I_q)_{max}\propto\omega_{rf}^2
            \label{w^2}
        \end{equation}
        该公式仅适用于未激发O波时的情况。
    \end{enumerate}
\end{enumerate}
\begin{figure}
    \centering
    \includegraphics[width=0.5\textwidth]{magnetic beach effect.png}
    \caption{Electron cyclotron heating by transverse acceleration of electrons in a magnetic field: “magnetic beach effect.”}
    \label{f-1}
\end{figure}

考虑右旋的呼啸波,共振吸收现象发生在衰减的磁场中,也被称为磁滩(magnetic beach),这与倾斜在海滩上的水消散的情况类似。如图\ref{f-1}所示。

对于截止波,会转化为其他形式的波或者被等离子体反射,或者被部分吸收。最理想的波还是右旋呼啸波,拥有与电子回旋方向一致的右旋圆极化(RCHP)电场。这个波会将馈入的微波能量传递给共振区的电子。

由于在截止点或者被腔壁反射的微波功率可能不会被回旋共振吸收,但是也有可能通过上混杂形式共振的X波吸收$\omega_{rf}=\omega_{UH}$。

\subsection{宏观描述:本征模式}
ECRIS中的微波也可以被宏观描述,比如考虑等离子体腔是一个多模腔,里面可以存在许多种电磁波的模式,并且可以通过微波耦合系统选择何种模式。圆柱形腔体(ECRIS的弧腔)的本征模为$\text{TE}_{m,n,p}$或$\text{TM}_{m,n,p}$,这些角标分别代表周向$\Psi$,径向$r$,和轴向$z$。\textbf{它们实际上是相对平行的波形成的驻波},在真空中的色散关系为
\begin{equation}
    k_0^2=k_\perp^2(m,n)+k_\parallel^2(p)
\end{equation}
其中$k_0$是真空中的波数。

在理想的情况中,ECRIS中的微波耦合表现为注入的微波只有右旋波可以耦合(平行传播的呼啸波),并且在其穿过共振面时被完全吸收。

但是在真实情况中,被注入的线极化波沿主轴平行传播,然后有可以转化为呼啸波。但是只有很少的右旋波可以被吸收,左旋的微波功率则无法被完全吸收。因此大量未被吸收的微波在ECR面处被反射,随后被弧腔壁反射。所以,\textbf{等离子体的空腔模式可能存在耦合}。

让我们考虑Caprice的ECRIS,它的设计频率为10 GHz。弧腔的长度为160 mm ,直径为66 mm。计算得到它在9.5到10.5 GHz频率下的存在的模式如图\ref{f-2}所示。
\begin{figure}
    \centering
    \includegraphics[width=1\textwidth]{possible modes.png}
    \caption{Possible modes in vacuum for a Caprice ECRIS between 9.5 GHz and 10.5 GHz}
    \label{f-2}
\end{figure}

尽管存在简并的微波模式,但是可以通过微波耦合系统来选择特定的模式以避免电场的简并。最好的模式为TE$_{1,n,p}$和TM$_{m,n,p}$。

\section{ECRIS中应用的耦合方案}
大多数ECRIS的弧腔尺寸仅为微波波长的几倍,等离子体体积相对于弧腔来说也十分小。\textbf{因为等离子体具有吸收性},许多本征模叠加并且在同一频率下同时存在。但是这些模式并不具有相同的权重,因此弧腔中的电磁场分布主要取决于耦合系统的选择。

\subsection{同轴耦合}
为了紧凑的考虑,B. Jacquot利用同轴的微波发射器在Caprice型的离子源(1987)。从那之后,许多紧凑的ECRIS都采用同样的微波耦合方案。

但是看起来等离子体弧腔中存在的微波模式为$TM_{0,m,n}$。对于这种模式,电场的轴向分量为0,也就是在等离子体的主体区域,如图\ref{TEM}所示。
\begin{figure}
    \centering
    \includegraphics[width=0.2\textwidth]{TEM.png}
    \caption{TEM waves. Solid lines, electric lines: dashed lines, magnetic lines.}
    \label{TEM}
\end{figure}

而且,通过测量Caprice中轴对称等离子体的抗磁效应,等离子体的总能量为$nE$。事实上,根据MHD平衡定理,压力梯度通过抗磁电流的洛伦兹力平衡。这个抗磁电流产生的磁场与主磁场相反:
\begin{equation}
    \Delta P=j\times B
\end{equation}
一个围绕在等离子体的环可以测量它内部任意的磁通量变化。这个变化可以转化为电压:
\begin{equation}
    \int V dt=-N\Delta \Phi\approx\frac{1}{2}N\pi r^2B\frac{\Sigma nkT}{B^2/2\mu_0}
\end{equation}
其中$N$是环的圈数。抗磁性反应的是等离子体的总能量。

在Caprice中的抗磁性测量表明等离子体吸收的微波功率约为馈入功率的20$\%$,如图\ref{Efficiency-1}所示。
\begin{figure}
    \centering
    \includegraphics{Efficiency of the.png}
    \caption{Efficiency of the coaxial coupling into a Caprice ECRIS}
    \label{Efficiency-1}
\end{figure}

该系统可以在模式转化处产生驻波,因此可以通过垂直注入的方式避免方波导转为同轴线时的功率损耗。此外,当微波在同轴线中传播时,他们会在进入等离子体弧腔前经过一个共振面。如果此处的气压过高,那么会有一部分微波功率会被吸收。虽然这种方式对于紧凑型ECRIS是有利的,但是这种微波耦合方法是难以使用和低效率的。

为了提高紧凑的微波耦合系统的优势,可以采用(0,0)相控矩形双脊波导在实现同轴线中TE$_{1,1}$模式的传输。双脊波导中的电场在波导中逐渐转变为TE$_{1,1}$模式,如图\ref{Excitation-1}所示。在等离子体弧腔中,微波模式会转变为TE$_{1,n,p}$模式。这种耦合方法已经被模拟计算后应用到Caprice离子源中。在离子源的轴线中,\textbf{对于待电离的元素仍保持自由。}

\begin{figure}
    \centering
    \includegraphics{Excitation of.png}
    \caption{Excitation of the TE$_{1,1}$ mode of the coaxial waveguide by using a (0,0) phased
    rectangular waveguide bijunction}
    \label{Excitation-1}
\end{figure}

\subsection{矩形波导耦合}
另一种向等离子体辐射微波的方案是通过矩形波导在弧腔的注入端注入。利用这种方案的形式的波导,使用的TE$_{10}$,如图\ref{TE10}所示。即使这种类型的耦合方式效率很高,但是单波导耦合的选择性并不够,建议使用使用双波导耦合系统,有时也称为\textbf{“bijunction”}。为了\textbf{保持等离子体主轴的可达性(对于气体和金属元素的注入系统)},可以使用不同的射频耦合系统。

TE$_{1,n,p}$模式可以实现呼啸波高效的耦合到ECR弧腔中。双波导的微波相位是同相的,可以利用商用分离器实现。

TM$_{0,n,p}$并不是ECRIS中最好的模式,因为它的轴向电场分量不为0,,除此以外它的磁场分量是平行和垂直于主磁场的,这样会导致电子的损失。所以TM$_{0,n,p}$模式相比利用呼啸波的模式效率更低。

\textbf{利用TE$_{0,n,p}$模式在H面上(0,$\pi$)注入。两个(0,$\pi$)相位的波导通过3db的矩形波导耦合一个$\pi /2$相位转换器耦合到另一个波导。这个模式会很容易耦合呼啸波,但是它的横向电场在主轴上为0,相比TE$_{1,n,p}$模式更低效率。}

根据上述的分析TE$_{1,n,p}$模式似乎是最好的ECRIS耦合方式。但是一个源体的设计还需要考虑等离子体的形状,它是一个三星形状,同时还需要放置一个或者多个炉口和气口。为了避免波导管的出气效应,必须将波导放置在远离三星的位置上。

\begin{figure}
    \centering
    \includegraphics[width=0.3\textwidth]{TE10.png}
    \caption{TE$_{10}$ mode commonly used in ECRIS (arrows presents the electric field intensity
    and direction)}
    \label{TE10}
\end{figure}

\subsection{双频的使用}
在BLB的ECRIS中采用10GHz和14GHz的双频微波的加热方式实现了极大的效率提升。电子沿着磁力线运动,那些在通过第一个共振面后将要损失在等离子体弧腔壁上的电子,可能在通过第二个共振面后被约束。

多频加热似乎可以导致离子产生时间的缩短,并使电荷态分布向高电荷态移动。

\subsection{频率超过20 GHz}

当微波频率在20 GHz以下时,常常采用矩形波导进行微波传输。但是超过20 GHz后,高功率微波源常常采用回旋管,它的输出常为圆波导或者准光学传输,比如28 GHz微波采用TE$_{02}$的过尺寸波导。从微波源到离子源的微波传输线必须经过严格的计算和设计,以避免传输过程中出现模式转换而导致的微波损失。这些杂散的模式由于波导制作的不完美导致的,比如倾斜、弯曲、直径变化、高压绝缘和真空窗等。如图\ref{f-3}所示为28 GHz微波传输线。它可以提供两种微波馈入方式。在回旋管微波窗处安装了一个弧光探测器,用以防止打火对回旋管造成损伤。两个功率计分别测量输出功率和反射功率。模式过滤器用来过滤所有超过TE$_{02}$的模式,随后TE$_{02}$模式转化为TE$_{01}$模式,因为TE$_{01}$模式在长距离的传输中有更小的欧姆损失。随后TE$_{01}$模式可以转化为TE$_{11}$或HE$_{11}$模式,然后通过绝缘传输线馈入ECRIS中。

\begin{figure}
    \centering
    \includegraphics[width=0.8\textwidth]{Schematic diagram.png}
    \caption{Schematic diagram of a 28 GHz transmission line with two coupling possibilities(arrows indicate the electric field).}
    \label{f-3}
\end{figure}

在传输线中,波导的弯头是一个非常重要的器件,因为相速度接近的模式之间转换,通常发生在具有轴向曲率的过尺寸波导中。DC-break通常采用叠片铝和聚四氟乙烯环做成的半周期结构。

在核聚变研究中超过30 GHz频率的微波通常采用准光学方式。

\subsection{总结}
通常来说,在设计像ECRIS这样的设备时很难适应理论,特别在设计微波辐射器时,比如在径向上留出空间放置波导就会减弱径向的磁场约束。幸好ECRIS是一个多模腔,并且具有很复杂的磁场可以使电子在腔中总有可以获得横向能量的地方。但是入射的微波功率,几乎有一半通过不同的方式被损失:
\begin{enumerate}
    \item [1.]损失在波导和匹配系统中;
    \item [2.]进入到损失锥中的电子;
    \item [3.]反射功率;
    \item [4.]损失在腔壁上。
\end{enumerate}

ECRIS中总共的吸收功率可以写为:
\begin{equation}
    P_{abs}\approx\frac{n_ekT_eV}{\tau_e}
\end{equation}
其中$\tau_e$是电子的约束时间,$V$是热等离子体体积。

为了有效的传输和耦合微波功率到ECRIS中,需要在设计微波传输线时特别注意以最小化损耗。当超过1 kW的微波功率时,十分推荐使用过尺寸的波导管。但是波导的位置需要特别小心的选取,以防止寄生共振造成的出气效应。

\section{28 GHz耦合实例}
之前的推导中给出的结论是引出的流强与微波频率的平方成正比,如式\ref{w^2}所示。当考虑SERSE ECRIS在三个不同频率下获得数据来检验频率的scaling law。得到下表:
\begin{figure}
    \centering
    \includegraphics[width=0.8\textwidth]{Xe.png}
    \caption{Xe$^{20+}$ Beam Intensity Obtained with SERSE ECRIS}
\end{figure}
从表中可以看出28 GHz的效率是最低的。在实验中,28 GHz微波通过圆波导的TE$_{01}$模式进行耦合的,14.5 GHz和18 GHz微波通过矩形波导馈入。

考虑流强是馈入功率的函数,为了获得90$\mu \text{A}$的流强,18 GHz微波需要1800 W,而28 GHz微波在4 kW下获得了215$\mu \text{A}$的流强。此时的功率效率为50 nA/W

\bibliographystyle{alpha}
\bibliography{sample}

\end{document}