\documentclass[fontset=windows]{article}

% Language setting
% Replace `english' with e.g. `spanish' to change the document language
\usepackage[english]{babel}

% Set page size and margins
% Replace `letterpaper' with`a4paper' for UK/EU standard size
\usepackage[letterpaper,top=2cm,bottom=2cm,left=3cm,right=3cm,marginparwidth=1.75cm]{geometry}
\usepackage{caption2}
\usepackage{subfigure}
\usepackage{float}

% Useful packages
\usepackage{amsmath}
\usepackage{graphicx}
\graphicspath{{Figures/}}%文章所用图片在当前目录下的 Figures目录
\usepackage[colorlinks=true, allcolors=blue]{hyperref}
\usepackage{ctex}
\usepackage{bm}
\usepackage{gensymb}
\renewcommand{\figurename}{图}
\renewcommand{\tablename}{表}
\newtheorem{defination}{定义}[section]
\newtheorem{theorem}{定理}[section]
\newtheorem{lemma}[theorem]{引理}
\newtheorem{corollary}[theorem]{推论}

\title{ECR离子源阅读笔记}
\author{X.Y. Wang}

\begin{document}
\maketitle

\begin{abstract}
RF Heating in Electron Cyclotron Resonance Ion Sources 文章阅读笔记,争取在4月5日前完成。
\end{abstract}

\section{简介}
为了更好的理解ECRIS的原理,需要使用原子物理的相关知识,总结起来是以下的公式:
\begin{equation}
    n_e\tau_e=\frac{1}{S_{q,q-1}-S_{q+1,q}}
    \label{e-1}
\end{equation}
\begin{equation}
    S_{q,q-1}=\frac{1}{n_e}\int_{\varepsilon_{q,q-1}}^{\infty}\sigma_{q,q-1}\sqrt{\frac{2\varepsilon}{m_e}}F(\varepsilon)d\varepsilon
    \label{e-2}
\end{equation}
其中$\tau_q$是电荷态为$q$离子的寿命, $S_{q,q-1}$是电离率(它取决于电离截面$\sigma$), $F(\varepsilon)$是电子的能量分布,其中$\varepsilon$是电荷态的电离能。

式\ref{e-1}中的分母在低温时逐渐升高,但是当温度超过20-30 keV时下降。当电离率为定值时,只有产生率$n_e\tau_e$足够高时才能产生足够高电荷态的离子。这时引出的粒子流强为:
\begin{equation}
    I_q^z\approx\frac{1}{2}\frac{n_q^zqeV_{ex}}{\tau_{q,l}^z}
\end{equation}
其中$V_{ex}$是等离子体内核的体积,$n_q$是某电荷态离子的密度。这个公式表示离子密度和离子寿命是高性能ECRIS的关键。电子温度同样也很重要,但是过热的加热需要被避免,因为超热电子产生硬韧致辐射,这会增加超导磁铁低温恒温器的热载,并且对绝缘器件造成损伤。

以往对等离子体起弧和稳定性问题的研究是基于半经验公式。最小B结果是磁流体稳定的,它的磁压比$\beta\ll1$。基于$n_e=n_c=\frac{m\omega_{RF}^2\varepsilon_0}{e^2}$的假设,磁流体稳定性条件为:
\begin{equation}
    (\frac{B}{B_{ECR}})^2>2\cdot10^2kT_e\frac{\mu_0\varepsilon_0}{m_e}
\end{equation}
通常情况下$\frac{B}{B_{ECR}}\geq2B_{ECR}$,$B_{inj}\approx3\cdot B_{ECR}$或更多,$B_{ext}\approx B_{rad}$,$0.3<\frac{B_{min}}{B_{rad}}<0.45$。考虑电磁场的截止现象,R. Geller提出了流强和平均电荷态关于$\omega_{RF}$的scaling law:
\begin{align}
    I\propto\frac{\omega_{RF}^2}{M}
    <q>\,\propto\, log\omega_{RF}^{3.5}
\end{align}

根据这些scaling laws,ECRIS性能的提升也取决于微波频率和约束磁场的提升。发展更经济有效的磁铁,高效的微波耦合,和更深入的微波加热机理研究是目前所需要的。

\section{ECRIS中的加热模型}
Lieberman等人利用单电子模型,考虑动量随机效应去解释无碰撞的ECRIS电子加热模型。模型中假设电子的运动提供了波-粒子相互作用的随机相位。在一个抛物线型的磁场$B=B_{min}(1+z^2/L^2)$,其中$L=(\Delta B/ B)^{-1}$


\bibliographystyle{alpha}
\bibliography{sample}

\end{document}